\documentclass{ectj}

\usepackage{amsfonts,amssymb,graphics,epsfig,verbatim,bm,latexsym,amsmath,url,amsbsy}

\newtheorem{theorem}{Theorem}
\newtheorem{assumption}{Assumption}
\newtheorem{proposition}{Proposition}
\newtheorem{corollary}{Corollary}
\newtheorem{lemma}{Lemma}
\newtheorem{example}{Example}
\newtheorem{remark}{Remark}

\renewcommand{\thesection}{\arabic{section}}
\renewcommand{\theequation}{\arabic{section}.\arabic{equation}}
\renewcommand{\thetheorem}{\arabic{section}.\arabic{theorem}}
\renewcommand{\theassumption}{\arabic{section}.\arabic{assumption}}
\renewcommand{\theproposition}{\arabic{section}.\arabic{proposition}}
\renewcommand{\thecorollary}{\arabic{section}.\arabic{corollary}}
\renewcommand{\thelemma}{\arabic{section}.\arabic{lemma}}
\renewcommand{\theexample}{\arabic{section}.\arabic{example}}
\renewcommand{\theremark}{\arabic{section}.\arabic{remark}}

\newcounter{bean}
\newcounter{beana}
\newcounter{beanb}

\year 2017
\received{February 2017}
\accepted{November 2017}
\volume{20}
  
\setcounter{page}{1}

\title[Short-title]{Econometrics journal style guide\\
                    instructions for authors}

\author[Short-authors]{Author~B.~Author$^{\dagger}$ and
                        Another~N.~Other$^{\ddagger}$}

\address{$^{\dagger}$Focal Image Ltd, Wessex House,
                    Teign Road, Newton Abbot,
                    Devon, TQ12 4AA, UK.}
\email{info@focalimage.com}

\address{$^{\ddagger}$Mobile Ltd, Lanchester Hall,
                    Long Street, Zennor,
                    Cornwall, SI4 8JK, UK.}
\email{gen@mobile.co.uk}

\def\AmSTeX{$\cal A$\kern-.1667em\lower.5ex\hbox{$\cal M$}\kern-.125em
            $\cal S$-\TeX}
\def\BibTeX{{\rm B\kern-.05em{\sc i\kern-.025em b}\kern-.08em
            T\kern-.1667em\lower.7ex\hbox{E}\kern-.125emX}}

  \begin{document}

  \begin{abstract}
    This document explains how authors should use the {\it Econometrics~Journal}
    {\LaTeX} styles in order to submit CRC versions of their papers.
    The file {\tt ectj.cls} encapsulates the most important aspects of
    the style and should be used.
    Also provided is the {\BibTeX} style file, {\tt chicago.bst}, that
    can be used to generate references to style automatically.
    This document is not meant to replace the standard {\LaTeX}
    reference book by Lamport, which all authors should be familiar with before proceeding.

  \keywords{{\LaTeX}, Mathematics, {\TeX}, Typesetting.}

  \end{abstract}


  \section{Introduction}

    To use these files we assume that you have a basic {\TeX} installation
    (including the necessary files to run {\LaTeX}). Along with this file
    ({\verb|EctJ_Style_Instructions.TeX|}) and the PostScript graphics file {\tt fig01.eps}
    you should also have received:
%
    \begin{enumerate}
      \item {\tt ectj.cls}---{\LaTeX $2 _{\textstyle \varepsilon}$} class file.
      \item {\tt chicago.bst}---{\BibTeX} style file for references.
    \end{enumerate}
%
    {\em Please note that both of these files are plain ASCII files.}

    These files should be placed in the {\TeX} search directory where they will be
    picked up automatically.

    Of course, {\tt ectj.cls} is {\LaTeX $2 _{\textstyle \varepsilon}$} and this
    means that these files can also be used under {\LaTeX $2 _{\textstyle \varepsilon}$}. \\

    \textbf{NB.} Please use UK spelling throughout (e.g.\ centre, behaviour, analyse, modelling).  The only exception
     should be for proper nouns, e.g. Institute for the Study of Labor.  Note that the journal uses the ize/iza/izi spelling for words such as
     parametrize, realization and minimizing.\\

    \textbf{NB.} Latin terms should be in italics, e.g. \textit{a priori}, \textit{ex ante}. \\

    \textbf{NB.} Abbreviations such as cf., e.g., i.e., etc. should be in roman typeface. \\

    A number of common stylistic errors are listed at the end of this document.


  \section{Template}

    The mark-up of documents must conform to the following standard
    {\LaTeX} layout:

      \begin{tabbing}
       \verb|\documentclass{ectj}|\\
       $\langle{\it preamble}\rangle$\\
       \verb|\begin{document}|\\
       $\langle{\it main~body}\rangle$\\
       \verb|\end{document}|\\
      \end{tabbing}

  \subsection{$\langle{\it preamble}\rangle$}

    The commands that appear here are to do with the make-up of
    the title page. The following macros are available to change publication
    details.
      \begin{verbatim}
       \year 2017
       \received{February 2017}
       \accepted{November 2017}
       \volume{20}
       \setcounter{page}{1}
      \end{verbatim}
    \noindent First, the year is set and a received date and accepted date added. Next the
    volume number is given and finally the starting page number.

    Further macros are needed for the article title, author names and their affiliations. \\

    Indeed, these instructions use the following coding:
      \begin{verbatim}
       \title[Short-title]{Econometrics Journal Style Guide}
       \author[Short-authors]{Author~B.~Author$^{\dagger}$ and
                                Another~N.~Other$^{\ddagger}$}
       \address{$^{\dagger}$Focal Image Ltd,
                            Wessex House, Teign Road,
                            Newton Abbot, Devon,
                            TQ12 4AA, UK}
       \email{info@focalimage.com}

       \address{$^{\ddagger}$Mobile Ltd, Lanchester Hall,
                    Long Street, Zennor,
                    Cornwall, SI4 8JK, UK.}
       \email{gen@mobile.co.uk}
      \end{verbatim}
    \verb|Short-authors| and \verb|Short-title| are the text that appears
    in the running headers.

    One or more \verb|\author{...}| declarations can be given, similarly
    a \verb|\address{...}| and \verb|\email{...}| can be given for each
    \verb|\author|. \\

    \textbf{NB.} Each author should provide no more than \textbf{two} affiliations. \\

    \textbf{NB.} Where two or more authors belong to the same institution then the affiliation should
     only be listed once with author e-mail addresses appearing on the same line separated by commas.
      \begin{verbatim}
       \title[Short-title]{Econometrics Journal Style Guide}
       \author[Short-authors]{Author~B.~Author$^{\dagger}$ and
                                Another~N.~Other$^{\dagger}$}
       \address{$^{\dagger}$Focal Image Ltd,
                            Wessex House, Teign Road,
                            Newton Abbot, Devon,
                            TQ12 4AA, UK}
       \email{aba@focalimage.com, ana@focalimage.com}
      \end{verbatim}

    Please note that {\em all\/} definitions should be placed in
    the preamble before the\break\verb|\begin{document}| statement.
    This makes it much easier to see the extent of a macro and
    will speed up the processing of an accepted paper.

  \subsection{$\langle{\it main~body}\rangle$}

    The main body is actually made up of several sections.
    The initial text should be a self-contained summary of no more than 150 words, coded as follows:
      \begin{verbatim}
       \begin{abstract}
        Abstracts are meant to be give a brief flavour of the article.
        \ldots\ something here just to end the sentence.

       \keywords{Abstracts, Keywords, \ldots .}
       \end{abstract}
      \end{verbatim}
    Keywords may be added using the \verb|\keywords{...}| macro within
    the {\sf abstract} environment, which gives a list of words set
    in italics. Each new keyword should start with an initial capital letter. This code produces
      \begin{abstract}
       Abstracts are meant to be give a brief flavour of the article.
       \ldots\ something here just to end the sentence.

      \keywords{Abstracts, Keywords, \ldots, .}
      \end{abstract}

  \subsubsection{Headings}

    After the keywords we begin with the headings which are used to
    introduce each topic.
      \begin{center}
       \begin{tabular}{lll}
        A level heading & \verb|\section{...}|       & $\rm11~pt/13~pt$ {\sc Small Caps}, displayed.\\
        B level heading & \verb|\subsection{...}|    & $\rm10~pt/12~pt$ {\it italic}, displayed.    \\
        C level heading & \verb|\subsubsection{...}| & $\rm10~pt/12~pt$ {\it italic}, run-on.       \\
       \end{tabular}
      \end{center}

  \subsubsection{Footnotes}

    Footnotes may be produced using the \verb|\footnote{...}| command. \\

    \textbf{NB.} Footnote numbers should always be placed in the text at the end of the
     relevant sentence immediately after the full stop.\footnote{An illustrative footnote.}
     This footnote was coded with:

      \begin{verbatim}
       \footnote{An illustrative footnote.}
      \end{verbatim}

    \textbf{NB.} Footnote numbering should never appear within Theorems,
     Propositions, Lemmata etc. but be placed at the end of the preceding sentence. \\

    Footnotes may, however, be placed within Examples, Remarks etc. as described above.

  \subsubsection{Lists}

    Lists should generally be avoided. Algorithms, procedures etc. are dealt with separately; see Section 2.2.7. \\

    \textbf{NB.} Bullet point lists should be strictly avoided. \\

    \textbf{NB.} Lists within text should first use (a), (b) etc. then (i), (ii) etc. \\

    A list of items can be produced using the standard {\LaTeX} \texttt{list} environment. \\

    Lists should only be ordered numerically or alphabetically. \\

    For example, the enumerated list

      \setcounter{bean}{0}
      \begin{list}
       {(\alph{bean})}{\usecounter{bean}}
       \item First item.
        \setcounter{beana}{0}
        \begin{list}
         {(\roman{beana})}{\usecounter{beana}}
         \item First sub-item.
          \setcounter{beanb}{0}
          \begin{list}
           {\arabic{beanb}.}{\usecounter{beanb}}
           \item First subsub-item.
           \item Second subsub-item.
          \end{list}
         \item Second sub-item.
        \end{list}
       \item Second item.
       \item Third item.
      \end{list}

    was coded with:

      \begin{verbatim}
       \setcounter{bean}{0}
       \begin{list}
        {\arabic{bean}.}{\usecounter{bean}}
        \item First item.
         \setcounter{beana}{0}
         \begin{list}
          {(\alph{beana})}{\usecounter{beana}}
          \item First sub-item.
           \setcounter{beanb}{0}
           \begin{list}
            {(\roman{beanb})}{\usecounter{beanb}}
            \item First subsub-item.
            \item Second subsub-item.
           \end{list}
          \item Second sub-item.
         \end{list}
       \item Second item.
       \item Third item.
      \end{list}
      \end{verbatim}

    Note that the counters \texttt{bean}, \texttt{beanb} and \texttt{beanb} were declared in the preamble by
    the \verb|\newcounter{...}| command.


  \subsubsection{Maths}

    For examples on the coding of mathematics in {\TeX} see the many
    excellent books on the topic, e.g., \cite{tex}, \cite{lamport}.

    Simple displayed equations are formatted as follows:
    \[
      \sum_{i=1}^{n} i = \frac{n(n+1)}{2}
    \]
    where the coding used was
      \begin{verbatim}
       \[
       \sum_{i=1}^{n} i = \frac{n(n+1)}{2}
       \]
      \end{verbatim}
    Note that the equations are centred with alignment around the
    equals sign for multi-line equations, as can be seen in the next
    example
      \begin{eqnarray}
       \sum_{i=1}^{n} i & = & 1+2+\cdots+n\nonumber\\
                        & = & \frac{n(n+1)}{2}
      \end{eqnarray}
    The coding used for this example was:
      \begin{verbatim}
       \begin{eqnarray}
        \sum_{i=1}^{n} i & = & 1+2+\cdots+n\nonumber\\
                         & = & \frac{n(n+1)}{2}
       \end{eqnarray}
      \end{verbatim}

    The notation for combinations, ${n \choose r}$, can be coded using
    \verb|${n \choose r}$|. \\

    Equations are numbered within sections as above. The command
    \verb|\setcounter{equation}{0}| should be inserted immediately after
    the beginning of each section. \\

    The extra {\AmSTeX} symbols may be used and are loaded automatically
    via the {\tt amsfonts.sty} and  {\tt amssymb.sty} style files.

  \subsubsection{Notational Conventions}

    Standard statistics such as $t$- and $F$-statistics should be italicized and are coded, e.g., as \verb|$t$-statistic| or \verb|\textit{t}-statistic|. \\

    Try to avoid en or em rules as punctuation; replace as necessary using commas, brackets or semi-colons.  \\

    Superscripts and subscripts, e.g., $X_{j}^{i}$, are in equation mode
    \[
      X_{j}^{i}
    \]
    where the coding used was
      \begin{verbatim}
       \[
         X_{j}^{i}
       \]
      \end{verbatim}

    Convergence in probability $\stackrel{p}{\rightarrow}$, almost sure convergence $\stackrel{a.s.}{\rightarrow}$, convergence in distribution $\stackrel{d}{\rightarrow}$ and weak convergence $\Rightarrow$ are in equation mode respectively
    \[
      \stackrel{p}{\rightarrow}, \hspace{0.1cm}\stackrel{a.s.}{\rightarrow},
      \hspace{0.1cm} \stackrel{d}{\rightarrow} \hspace{0.1cm}
      \text{and} \hspace{0.1cm} \Rightarrow
    \]
    coded as
      \begin{verbatim}
       \[
         \stackrel{p}{\rightarrow},
         \hspace{0.1cm}\stackrel{a.s.}{\rightarrow},
         \hspace{0.1cm} \stackrel{d}{\rightarrow} \hspace{0.1cm}
         \text{and} \hspace{0.1cm} \Rightarrow
       \]
      \end{verbatim}

    Small and large order in probability, e.g., $o_{p}(n^{-\delta})$ and $O_{p}(n^{-\delta})$ respectively, are in equation mode
    \[
      o_{p}(n^{-\delta}) \hspace{0.1cm} and \hspace{0.1cm} O_{p}(n^{-\delta})
    \]
    where the coding used was
      \begin{verbatim}
       \[
         o_{p}(n^{-\delta}) \hspace{0.1cm} \text{and}
         \hspace{0.1cm} O_{p}(n^{-\delta})
       \]
      \end{verbatim}

    The expectation operator $E[\cdot]$ uses the bracket convention $[\cdot]$ and is coded as \verb|$E[\cdot]$|. \\

    \textbf{NB.} Bracket sizes should generally not be cascaded; bracket sizes should only match the size
    of the adjacent text for fractions and matrices. Try to avoid using the \LaTeX commands \verb|\left| and \verb|\right| to
    size brackets. Instead, find the correct size using \verb|\big|, \verb|\Big|, etc. \\

    \textbf{NB.} Sequences of brackets $((\ldots))$ coded as \verb|$((\ldots))$| are preferred in mathematical text
     to the bracket convention \verb|${[(\ldots)]}$|.



  \subsubsection{Assumptions, Corollaries, Definitions, Lemmata, Propositions and Theorems}

    Assumptions, Corollaries, Definitions, Lemmata, Propositions and Theorems require the
    appropriate environment to be defined. For example, the command

      \begin{verbatim}
       \newtheorem{theorem}{Theorem}
      \end{verbatim}

    \noindent included after \verb|\usepackage{ ... }| at the beginning of
    the document creates the environment for Theorems. The additional command

      \begin{verbatim}
       \renewcommand{\thetheorem}{\arabic{section}.\arabic{theorem}}
      \end{verbatim}

    \noindent also included at the beginning of the document ensures Theorems
    are numbered by and within section. The ~\verb|theorem|~ counter is re-initialized for
    each section by insertion of the command \verb|\setcounter{theorem}{0}|
    immediately after the beginning of each section.

     \begin{theorem}
     \label{th-limdistr}
     If Assumptions 2.1 and 2.2 are satisfied then
     \begin{equation}
      n^{1/2}
      \bigg( \begin{array}{c}
       \hat{\beta}-\beta_{0} \\
       \hat{\lambda}
      \end{array} \bigg)
      \stackrel{d}{\rightarrow}N(0,{\rm diag}(\Sigma ,P)),
       2n\Big(\sum_{i=1}^{n}\rho(\hat{\lambda}^{\prime }
       g_{i}(\hat{\beta}))/n-\rho _{0}\Big) \stackrel{d}{\rightarrow }\chi ^{2}(m-p),
     \end{equation}
     where $\Sigma \equiv (G^{\prime }\Omega ^{-1}G)^{-1}$ and
      $P\equiv\Omega ^{-1}-\Omega^{-1}G\Sigma G^{\prime }\Omega ^{-1}$.
    \end{theorem}

    \noindent was coded as

      \begin{verbatim}
       \begin{theorem}
        \label{th-limdistr}
        If Assumptions 2.1 and 2.2 are satisfied then
        \begin{equation}
         n^{1/2}
         \bigg( \begin{array}{c}
          \hat{\beta}-\beta_{0} \\
          \hat{\lambda}
         \end{array} \bigg)
         \stackrel{d}{\rightarrow}N(0,{\rm diag}(\Sigma ,P)),
          2n \Big(\sum_{i=1}^{n}\rho(\hat{\lambda}^{\prime }
          g_{i}(\hat{\beta}))/n-\rho _{0}\Big)
          \stackrel{d}{\rightarrow }\chi ^{2}(m-p),
        \end{equation}
        where $\Sigma \equiv (G^{\prime }\Omega ^{-1}G)^{-1}$ and
         $P\equiv\Omega ^{-1}-\Omega^{-1}G\Sigma G^{\prime }\Omega ^{-1}$.
       \end{theorem}
      \end{verbatim}

    The label command \verb|\label{th-limdistr}| allows Theorem~\ref{th-limdistr} to be
    cross-referenced by the use of the command \verb|\ref{th-limdistr}|. \\

    \textbf{NB.} Mathematical proofs that are an integral part of the paper's main narrative should be included in its main text, but all other proofs and subsidiary results should appear in its appendices or as Supporting Information. \\

    \textbf{NB.} When enumerated, Assumptions, etc., should be proper nouns, i.e., capitalized as, e.g., Assumption 2.1;
     otherwise, they should be written in lower case, e.g., assumptions. \\

    \textbf{NB.} Redundant text such as `Assume that (a) ...' in assumptions and `Then it holds that ...' and `Then ...' in theorems should be avoided. \\

    Lists within Assumptions, Corollaries, Definitions, Lemmata, Propositions and Theorems
    should first use (a), (b) etc., then (i), (ii) etc.; they should continue on the same line
    and should be formatted as follows:
     \begin{verbatim}
       \begin{assumption}
        (a) $\beta_{0} \in \mathcal{B}$ is the unique solution to
        $E[g(z,\beta )]=0$;   (b) $\mathcal{B}$ is compact;
        (c) (i)  $g(z,\beta )$ is continuous at each
        $\beta \in \mathcal{B}$ with probability one;
            (ii) $E[\sup_{\beta \in \mathcal{B}}\| g(z,\beta)\|^{\alpha }]
            < \infty$ for some $\alpha > 2$.
       \end{assumption}
      \end{verbatim}

\begin{assumption}
     (a) $\beta_{0} \in \mathcal{B}$ is the unique solution to $E[g(z,\beta )]=0$;
     (b) $\mathcal{B}$ is compact;
     (c) (i)  $g(z,\beta )$ is continuous at each $\beta \in \mathcal{B}$
              with probability one;
         (ii) $E[ \sup_{\beta \in \mathcal{B}} \| g(z,\beta )\| ^{\alpha }]  < \infty$ for some $\alpha > 2$.
    \end{assumption}

  \subsubsection{Algorithms, Examples and Remarks}

    Algorithms, Examples and Remarks also require the appropriate environment to be defined
    in a similar fashion to those discussed above for Theorems etc. However, the
    text in Algorithms, Examples and Remarks should \textbf{not} be italicized.

      \begin{verbatim}
       \begin{remark}
        \textnormal{From the above theorem, we notice that the asymptotic
        local power of the test based on $t^{+}$ depends only on the mean
        of $\theta _{i}$, $\mu _{\theta }$ and not on heterogeneity of the
        local alternative around this mean.}
       \end{remark}
      \end{verbatim}

    \noindent results in

      \begin{remark}
       \textnormal{From the above theorem, we notice that the asymptotic local power of the
       test based on $t^{+}$ depends only on the mean of $\theta _{i}$, $\mu _{\theta }$
       and not on heterogeneity of the local alternative around this mean.}
      \end{remark}

    Examples, Remarks, etc., are also numbered by and within section. Correspondingly their counters
    must be re-initialized at the beginning of each section as described above. \\

    \textbf{NB.} When enumerated, Examples, etc., should be proper nouns, i.e., capitalized as, e.g., Example 2.1;
     otherwise, lower case should be used, e.g., examples. \\

    Sequences of instructions required in algorithms, etc., should appear as follows.

      \setcounter{bean}{0}
      \begin{center}
       \begin{list}
       {\textsc{Step} \arabic{bean}.}{\usecounter{bean}}
         \item Initialize procedure.
         \item Perform calculation.
         \item Repeat calculations $N$ times.
       \end{list}
      \end{center}

    coded as

      \begin{verbatim}
       \setcounter{bean}{0}
       \begin{center}
        \begin{list}
         {\textsc{Step} \arabic{bean}.}{\usecounter{bean}}
         \item Initialize procedure.
         \item Perform calculation.
         \item Repeat calculations $N$ times.
        \end{list}
       \end{center}
      \end{verbatim}

    \textbf{NB.} When enumerated, the steps in Algorithms, etc., should be proper nouns, i.e., capitalized, e.g., as Step 1;
     otherwise, lower case should be used, e.g., steps. \\

  \subsubsection{Tables}

    Table~\ref{tab01} shows an output of the coding that follows.

\begin{table}[b]
       \caption{\label{tab01}Daily returns for Sterling.}
       \begin{center}
        \begin{tabular*}{\textwidth}{@{}lcccccc@{}}
         \hline\hline
         & Mean & MC S.E.\ & Inefficiency & \multicolumn{3}{c}{Covariance and {\it correlation}} \\
         \hline
         $\phi|y$ & 0.97762 & 0.00013754 & 163.55 & 0.00011062 &
            $-{\it 0.684}$ & {\it 0.203} \\
         $\sigma_{\eta} | y$ & 0.15820 & 0.00063273 & 386.80 & $-0.00022570$
            & 0.00098303 & $-{\it 0.129}$ \\
         $\beta | y$ & 0.64884 & 0.00036464 & 12.764 & 0.00021196
            & $-0.00040183$ & 0.0098569 \\ [6pt]
         Time & 5829.5 & 0.58295 &  &  &  &  \\
         \hline\hline
        \end{tabular*}
       \end{center}
       \footnotesize
       \renewcommand{\baselineskip}{11pt}
       \textbf{Note:} The Monte Carlo standard error of the simulation is computed
          using bandwidths of 2,000, 4,000 and 2,000, respectively. Italics
          are correlations rather than covariances of the posterior. Computer
          time is seconds on a Pentium Pro/200. The other time is the number
          of seconds taken to perform 100 sweeps of the sampler.
      \end{table}
      \clearpage

      \begin{verbatim}
       \begin{table}
        \caption{\label{tab01}Daily returns for Sterling.}
        \begin{center}
         \begin{tabular*}{\textwidth}{@{}lcccccc@{}}
          \hline\hline
          & Mean & MC S.E.\ & Inefficiency &
          \multicolumn{3}{c}{Covariance and {\it correlation}} \\
          \hline
          $\phi|y$ & 0.97762 & 0.00013754 & 163.55 & 0.00011062 &
            $-{\it 0.684}$ & {\it 0.203} \\
          $\sigma_{\eta} | y$ & 0.15820 & 0.00063273 & 386.80 & $-0.00022570$
            & 0.00098303 & $-{\it 0.129}$ \\
          $\beta | y$ & 0.64884 & 0.00036464 & 12.764 & 0.00021196
            & $-0.00040183$ & 0.0098569 \\ [6pt]
          Time & 5829.5 & 0.58295 &  &  &  &  \\
          \hline\hline
         \end{tabular*}
       \end{center}
       \footnotesize
       \renewcommand{\baselineskip}{11pt}
       \textbf{Note:} The Monte Carlo standard error of the simulation
          is computed using bandwidths  of 2,000, 4,000 and 2,000, respectively.
          Italics are correlations rather than covariances of the posterior.
          Computer time is seconds on a Pentium Pro/200. The other time
          is the number of seconds taken to perform 100 sweeps of the sampler.
       \end{table}
      \end{verbatim}

    \textbf{NB.} Avoid $\times10^{-a}$, i.e., use $0.0002$ rather than $2.0\times10^{-4}$. Rescale tables if necessary. \\

    \textbf{NB.} When enumerated tables should be proper nouns, i.e., capitalized as, e.g., Table 1;
     otherwise, lower case should be used, e.g., tables. \\

    \textbf{NB.} When a paper is typeset, where suitable, columns will be aligned on the decimal point.\\

    \textbf{NB.} Any simulation results may be concisely summarized in the main text, normally within one page using a key table or graph, but their details must be submitted as \textbf{Supporting Information}. \\

    Here \verb|\begin{table}| instructs {\TeX} that we are about
    to create a table. \verb|\caption{...}| creates a table caption with the appropriate
    number. \verb|\begin{center}| ... \verb|end{center}|
    aligns the table on the center of the page horizontally. \\

    \textbf{NB.} Table captions should only be the title for the table, be short (no longer than a line) and
     centred. Explanatory notes should be appended at the end of the table and should not be centred. The use
     of \verb|\footnotesize| places the text of the note in the smaller type of a footnote and
     \verb|\renewcommand{\baselineskip}{11pt}| ensures a smaller line spacing of \verb|11pt| height. \\

    \verb|\begin{tabular*}| has two extra arguments that specify how
    wide the final table should be as well as how each column is to be
    aligned. The alignment \verb|{@{}lcccccc@{}}| first removes any
    leading space, \verb|@{}|, and then makes the first column left-aligned.
    Next, we have six columns that are centred.

    Horizontal rules are provided by the \verb|\hline| which may be
    doubled up to produce a slightly thicker line.

    Each row of the table is then given with columns separated
    by an~\verb|&| and rows separated by~\verb|\\|.



  \subsubsection{Figures}

    To include figures the {\tt graphicx.sty} package has been preloaded,
    but others may be used, such as {\tt psfig.sty}, by adding an extra
    statement with the {\verb|\usepackage|} command. For example, the {\tt
    psfig.sty} macros will be made available if you add to the preamble of
    your file

      \begin{verbatim}
       \usepackage{psfig}
       ...
      \end{verbatim}

    Figures should be placed near to where they are first referred
    to. All figures should be supplied as {\tt .eps} PostScript files. \\

    \textbf{NB.} Figure captions should also be short and centred. Descriptions,
     explanatory notes and other details of figures should be given in the text. \\

    \textbf{NB.} When enumerated figures should be proper nouns, i.e., capitalized as Figure~1;
     otherwise, lower case should be used, e.g., figures. \\

    \textbf{NB.} Any simulation results may be concisely summarized in the main text, normally within one page using a key table or graph, but their details must be submitted as \textbf{Supporting Information}. \\

    The following coding will include your figures:
      \begin{verbatim}
       \begin{figure}
        \centering
         {\includegraphics[width=0.85\textwidth]{fig01.eps} }
         \caption{\label{fig01}QQ Plots: $\theta _{2}$ unknown.}
       \end{figure}
      \end{verbatim}

      \begin{figure}
       \centering
        {\includegraphics[width=0.5\textwidth]{fig01.eps} }
        \caption{\label{fig01}QQ Plots: $\theta _{2}$ unknown.}
      \end{figure}

    The effect of the above input can be seen in Figure~\ref{fig01}. Note that
      \begin{verbatim}
       \includegraphics[width=0.5\textwidth]{fig01.eps}
      \end{verbatim}
    sets the figure to be 50\% of the textwidth, but any percentage can be used. \\

    {\tt graphicx.sty} is part of the {\tt graphics} bundle in
    the {\LaTeX $2 _{\textstyle \varepsilon}$} distribution. Please read
    {\tt grfguide.ps} which is the documentation for graphic inclusion
    available in your distribution.

    \subsubsection{Acknowledgements}

    Acknowledgments should be placed at the end of the main body of the paper and
    immediately preceding the bibliography. For example,

      \section*{Acknowledgements}
       The authors are grateful to ... \\

    \noindent was coded with:

      \begin{verbatim}
      \section*{Acknowledgements}
       The authors are grateful to ...
      \end{verbatim}


  \subsubsection{References}

    References are Harvard style, i.e., `Author (Year) and Author
    (Year) investigated...' or `...; see, e.g.\ Author (Year) and Author (Year).'
    These are coded as \verb|\citet{...} and \citet{...} investigated...| and
    \verb|...; see, e.g.\ \citet{...}|
    \verb|and \citet{...}.|
    Try to avoid citing references within parentheses but, if necessary,
    use `(Author, Year, and Author, Year)', coded as
    \verb|(\citealp{...}, and \citealp{...})|.
    You can obtain the first by using \verb|\citet{...}|. You can use
    {\BibTeX} and the supplied {\tt chicago.bst} to generate references in the
    correct style for the journal. \\

    \textbf{NB.} References should first be ordered strictly alphabetically and
     then chronologically. Text citations to references should appear in alphabetical
     and chronological order. \\

    \textbf{NB.} In journal titles, any ampersand \verb|&| must be amended to `and'. Also, do not use `{\it The}' at the beginning of a journal title.\\

    \textbf{NB.} Citations in the text to references with three or more authors should use the last name
     of the first author and et al., e.g. Hausman et al. \\

    Alternatively a {\BibTeX} file {\tt references.bib} may be imported by
    inclusion of the command
      \begin{verbatim}
       \bibliography{references}
      \end{verbatim}
    immediately after the Acknowledgments section and before any
    Appendices. The command \verb|\bibliographystyle{chicago}| must be included
    and may be placed anywhere after \verb|\begin{document}|. \\

    \textbf{NB.} When supplying {\TeX} codes, we also need any {\tt .bbl}
    or {\tt .bib} files that you use as well. \\

    Within the {\sf thebibliography} environment you must use a modified form for
    each \verb|\bibitem|. Following each \verb|\bibitem| is the sequence
      \begin{verbatim}
       [\protect\citeauthoryear{full-author-info}{abbrev-author-info}{year}]
      \end{verbatim}
    which is then followed by the internal label for the reference, \verb|{ref-label}|.
    For articles with more than two authors \verb|{abbrev-author-info}| is written as
    \verb|{first-author et al.}| whereas for articles with one or two authors
    \verb|{author}| or \verb|{author and another}| should be used. Note that `et al.'
    is {not} italicized. \\

     \textsc{Books, texts, monographs} The place of publication and publisher must be stated for books, texts, monographs, etc.
     The state should always be given for US locations. Some examples are given here.\\

     \begin{verbatim}
       \begin{thebibliography}{}
        ...
        \bibitem[\protect\citeauthoryear{Basawa and Scott}{Basawa and Scott}
        {1983}]{BasawaScott}
           Basawa, I.~V. and D.~J. Scott (1983).
        \newblock {\em Asymptotic Optimal Inference for Non-ergodic Models\/}.
        \newblock Lecture Notes in Statistics, Volume 17. New York: Springer.

        \bibitem[\protect\citeauthoryear{Fuller}{Fuller}{1996}]{Fuller}
           Fuller, W.~A. (1996).
        \newblock {\em Introduction to Time Series\/} (2nd ed.).
        \newblock New York, NY: John Wiley.

        \bibitem[\protect\citeauthoryear{Gilks, Richardson, and Spiegelhalter}{Gilks
           et~al.}{1996}]{GilksRichardsonSpiegelhalter}
           Gilks, W.~K., S.~Richardson, and D.~J. Spiegelhalter (1996).
        \newblock {\em Markov Chain Monte Carlo in Practice}.
        \newblock London: Chapman and Hall.

        \bibitem[\protect\citeauthoryear{Knuth}{Knuth}{1986}]{tex}
           Knuth, D.~E. (1986).
        \newblock {\em The {\TeX}book}.
        \newblock Reading, MA: Addison-Wesley.

        \bibitem[\protect\citeauthoryear{Lamport}{Lamport}{1994}]{lamport}
           Lamport, L. (1994).
        \newblock {\em {\LaTeX}: A Document Preparation System User's Guide and
           Reference Manual}.
        \newblock Reading, MA: Addison-Wesley.

        \bibitem[\protect\citeauthoryear{R Core Team}{R Core Team}{2012}]{RCoreTeam}
           R Core Team (2012).
        \newblock {\em R: A Language and Environment for Statistical Computing}.
        \newblock Vienna: R Foundation for Statistical Computing.
        ...
       \end{thebibliography}
      \end{verbatim} \\


    \textsc{chapters, articles etc.} \\

    Page numbers for chapters or articles in books must be provided.
     The state should always be given for US locations. Some examples are given here.\\

         \begin{verbatim}
       \begin{thebibliography}{}
        ...
        \bibitem[\protect\citeauthoryear{Bollerslev, Engle, and Nelson}
           {Bollerslev et~al.}{1994}]{BollerslevEngleNelson}
           Bollerslev, T., R.~F. Engle, and D.~B. Nelson (1994).
        \newblock \uppercase{ARCH} models.
        \newblock In R.~F. Engle and D.~McFadden (Eds.), {\em Handbook of
           Econometrics, Volume IV}, 2959--3038. Amsterdam: North-Holland.

        \bibitem[\protect\citeauthoryear{Gourieroux and Monfort}{Gourieroux and
           Monfort}{1994}]{GourierouxMonfort}
           Gourieroux, C. and A.~Monfort (1994).
        \newblock Testing non-nested hypotheses.
        \newblock In R.~F. Engle and D.~McFadden (Eds.), {\em Handbook of
        Econometrics, Volume IV}, 2583--637. Amsterdam: North-Holland.

        \bibitem[\protect\citeauthoryear{Pedroni}{Pedroni}{2000}]{Pedroni}
           Pedroni, P. (2000).
        \newblock Fully modified OLS for heterogeneous cointegrated panels.
        \newblock In B. Baltagi (Ed.), {\em Nonstationary Panels, Panel
        Cointegration, and Dynamic Panels: 15 (Advances in Econometrics)},
        93--130. New York, NY: JAI Press.
        ...
       \end{thebibliography}
      \end{verbatim} \\


    \textsc{Working papers, reports etc.} \\

    For working papers or reports the series number and series, if available, and place
    of publication should always be given; if the place of publication is unknown the
    university or institutional affiliation of the first author should be given. Some examples are given here.\\

      \begin{verbatim}
       \begin{thebibliography}{}
        ...
        \bibitem[\protect\citeauthoryear{Borak, H\"{a}rdle, Mammen and Park}
        {Borak et~al.}{2007}]{Borak}
           Borak, S., W. H\"{a}rdle, E. Mammen and B.~U. Park (2007).
        \newblock Time series modelling with semi-parametric factor dynamics.
        \newblock Working paper, Humboldt University.

        \bibitem[\protect\citeauthoryear{Hausman et al.}
           {Hausman et al.}{2007}]{HausmanNeweyWoutersenChaoSwanson}
           Hausman, J.A., W.K. Newey, T. Woutersen, J. Chao and N. Swanson (2007).
        \newblock Instrumental variable estimation with heteroskedasticity and
           many instruments.
        \newblock CWP 22/07, Centre for Microdata Methods and Practice,
        Institute for Fiscal Studies and University College London.

        \bibitem[\protect\citeauthoryear{Moreira, Porter and Suarez}
        {Moreira et al.}{2007}]{Moreira}
           Moreira, M.~J., J.~R. Porter and G.~A. Suarez (2007).
        \newblock Bootstrap and higher-order expansion validity
        when instruments may be weak.
        \newblock Technical Working Paper 302, National Bureau of
        Economic Research (revised).
        ...
       \end{thebibliography}
      \end{verbatim} \\


    \textsc{Published papers, forthcoming papers etc.} \\
 The volume number of published journal articles and page numbers must be provided. Note the page ranges
 are separated by an en rule and the final page number does not repeat unnecessary hundreds.
 Issue numbers are not given unless the journal does not have continuously paginated volumes. The two
 rare exceptions for which issue numbers should be included are \textit{American Economic Review}
 and \textit{Journal of Economic Perspectives}.\\

 Some examples of journal references are given here.\\

      \begin{verbatim}
       \begin{thebibliography}{}
        ...
        \bibitem[\protect\citeauthoryear{Aitchison}{Aitchison}{1962}]{Aitchison}
           Aitchison, J. (1962).
        \newblock Large-sample restricted parametric tests.
        \newblock {\em Journal of the Royal Statistical Society\/},
        Series B {\em 69}, 234--50.

        \bibitem[\protect\citeauthoryear{Bollerslev}{Bollerslev}{1987}]{Bollerslev}
        Bollerslev, T. (1987).
        \newblock A conditional heteroskedastic time series model for speculative
        prices and rates of return.
        \newblock {\em Review of Economics and Statistics\/} {\em 69}, 542--47.

        \bibitem[\protect\citeauthoryear{Chib and Greenberg}{Chib and
           Greenberg}{1994}]{ChibGreenberg}
           Chib, S. and E.~Greenberg (1994).
        \newblock Bayes inference for regression models with ${\rm ARMA}(p,q)$
           errors.
        \newblock {\em Journal of Econometrics\/} {\em 64}, 183--206.

        \bibitem[\protect\citeauthoryear{Cox}{Cox}{1991}]{Cox(91)}
           Cox, D.~R. (1991).
        \newblock Long-range dependence, non-linearity and time irreversibility.
        \newblock {\em Journal of Time Series Analysis\/}~{\em 12}, 329--35.

        \bibitem[\protect\citeauthoryear{Hallin et al.}{Hallin et~al.}{2008}]
        {HallinVermandeleWerker}
           Hallin, M., C. Vermandele and B. Werker (2008).
        \newblock Semiparametrically efficient inference based on signs  and ranks
           for median-restricted models.
        \newblock {\em Journal of the Royal Statistical Society\/}, Series B {\em 70},
           389--412.

        \bibitem[\protect\citeauthoryear{Hansen}{Hansen}{2007}]{Hansen}
           Hansen, B. (2007).
        \newblock Uniform convergence rates for kernel estimation with dependent data.
        \newblock Forthcoming in {\em Econometric Theory\/}.

        \bibitem[\protect\citeauthoryear{Park and Miller}{Park and
           Miller}{1988}]{ParkMiller}
           Park, S. and K.~Miller (1988).
        \newblock Random number generators: good ones are hard to find.
        \newblock {\em Communications of the Association for Computing Machinery\/}
        {\em 31}, 1192--201.

        \bibitem[\protect\citeauthoryear{Pesaran and Pesaran}{Pesaran and
           Pesaran}{1993}]{PesaranPesaran}
           Pesaran, M.~H. and B.~Pesaran (1993).
        \newblock A simulation approach to the problem of computing {C}ox's
           statistic for testing nonnested models.
        \newblock {\em Journal of Econometrics\/} {\em 57}, 377--92.

        \bibitem[\protect\citeauthoryear{Phillips}{Phillips}{1991}]{Phillips}
           Phillips, P.~C.~B. (1991).
        \newblock To criticise the critics: an objective {B}ayesian
           analysis of stochastic trends.
        \newblock {\em Journal of Applied Econometrics\/} {\em 6}, 333--64.
        ...
       \end{thebibliography}
      \end{verbatim} \\


    Finally, the complete set of references should appear as follows. \\

    \begin{thebibliography}{}

     \bibitem[\protect\citeauthoryear{Aitchison}{Aitchison}{1962}]{Aitchison}
        Aitchison, J. (1962).
     \newblock Large-sample restricted parametric tests.
     \newblock {\em Journal of the Royal Statistical Society,\/} Series B {\em 69}, 234--50.

     \bibitem[\protect\citeauthoryear{Basawa and Scott}{Basawa and Scott}{1983}]{BasawaScott}
        Basawa, I.~V. and D.~J. Scott (1983).
     \newblock {\em Asymptotic Optimal Inference for Non-ergodic Models\/}.
     \newblock Lecture Notes in Statistics, Volume 17. Berlin: Springer.

     \bibitem[\protect\citeauthoryear{Bollerslev}{Bollerslev}{1987}]{Bollerslev}
        Bollerslev, T. (1987).
     \newblock A conditional heteroskedastic time series model for speculative
        prices and rates of return.
     \newblock {\em Review of Economics and Statistics\/} {\em 69}, 542--47.

     \bibitem[\protect\citeauthoryear{Bollerslev, Engle, and Nelson}{Bollerslev et~al.}{1994}]{BollerslevEngleNelson}
        Bollerslev, T., R.~F. Engle and D.~B. Nelson (1994).
     \newblock \uppercase{ARCH} models.
     \newblock In R.~F. Engle and D.~McFadden (Eds.), {\em Handbook of
        Econometrics, Volume 4}, 2959--3038. Amsterdam: North-Holland.

     \bibitem[\protect\citeauthoryear{Borak, H\"{a}rdle, Mammen and Park}{Borak et~al.}{2007}]
        {Borak}
        Borak, S., W. H\"{a}rdle, E. Mammen and B.~U. Park (2007).
     \newblock Time series modelling with semi-parametric factor dynamics.
     \newblock Working paper, Humboldt University.

     \bibitem[\protect\citeauthoryear{Chib and Greenberg}{Chib and
        Greenberg}{1994}]{ChibGreenberg}
        Chib, S. and E.~Greenberg (1994).
     \newblock Bayes inference for regression models with ${\rm ARMA}(p,q)$
        errors.
     \newblock {\em Journal of Econometrics\/} {\em 64}, 183--206.

     \bibitem[\protect\citeauthoryear{Cox}{Cox}{1991}]{Cox(91)}
        Cox, D.~R. (1991).
     \newblock Long-range dependence, non-linearity and time irreversibility.
     \newblock {\em Journal of Time Series Analysis\/}~{\em 12}, 329--35.

     \bibitem[\protect\citeauthoryear{Fuller}{Fuller}{1996}]{Fuller}
        Fuller, W.~A. (1996).
     \newblock {\em Introduction to Time Series\/} (2nd ed.).
     \newblock New York, NY:  Wiley.

     \bibitem[\protect\citeauthoryear{Gilks et al.}{Gilks et~al.}{1996}]{GilksRichardsonSpiegelhalter}
        Gilks, W.~K., S.~Richardson and D.~J. Spiegelhalter (1996).
     \newblock {\em Markov Chain Monte Carlo in Practice}.
     \newblock London: Chapman and Hall.

     \bibitem[\protect\citeauthoryear{Gourieroux and Monfort}{Gourieroux and
        Monfort}{1994}]{GourierouxMonfort}
        Gourieroux, C. and A.~Monfort (1994).
     \newblock Testing non-nested hypotheses.
     \newblock In R.~F. Engle and D.~McFadden (Eds.), {\em Handbook of
        Econometrics, Volume 4}, 2583--637. Amsterdam: North-Holland.

     \bibitem[\protect\citeauthoryear{Hallin et al.}{Hallin et~al.}{2008}]{HallinVermandeleWerker}
        Hallin, M., C. Vermandele and B. Werker (2008).
     \newblock Semiparametrically efficient inference based on signs  and ranks
        for median-restricted models.
     \newblock {\em Journal of the Royal Statistical Society\/}, Series B {\em 70},
        389--412.

     \bibitem[\protect\citeauthoryear{Hansen}{Hansen}{2007}]{Hansen}
        Hansen, B. (2007).
     \newblock Uniform convergence rates for kernel estimation with dependent data.
     \newblock Forthcoming in {\em Econometric Theory\/}.

     \bibitem[\protect\citeauthoryear{Hausman et al.}{Hausman et al.}{2007}]{HausmanNeweyWoutersenChaoSwanson}
        Hausman, J.~A., W.~K. Newey, T. Woutersen, J. Chao and N. Swanson (2007).
     \newblock Instrumental variable estimation with heteroskedasticity and many instruments.
     \newblock CWP 22/07, Centre for Microdata Methods and Practice, Institute for
        Fiscal Studies and University College London.

     \bibitem[\protect\citeauthoryear{Knuth}{Knuth}{1986}]{tex}
        Knuth, D.~E. (1986).
     \newblock {\em The {\TeX}book}.
     \newblock Reading, MA: Addison-Wesley.

     \bibitem[\protect\citeauthoryear{Lamport}{Lamport}{1994}]{lamport}
        Lamport, L. (1994).
     \newblock {\em {\LaTeX}: A Document Preparation System User's Guide and
        Reference Manual}.
     \newblock Reading, MA: Addison-Wesley.

     \bibitem[\protect\citeauthoryear{Moreira et al.}{Moreira et~al.}{2007}]{Moreira}
        Moreira, M.~J., J.~R. Porter and G.~A. Suarez (2007).
     \newblock Bootstrap and higher-order expansion validity when instruments may be weak.
     \newblock Technical Working Paper 302, National Bureau of Economic Research (revised).

     \bibitem[\protect\citeauthoryear{Park and Miller}{Park and
        Miller}{1988}]{ParkMiller}
        Park, S. and K.~Miller (1988).
     \newblock Random number generators: good ones are hard to find.
     \newblock {\em Communications of the Association for Computing Machinery\/} {\em 31}, 1192--201.

     \bibitem[\protect\citeauthoryear{Pedroni}{Pedroni}{2000}]{Pedroni}
        Pedroni, P. (2000).
     \newblock Fully modified OLS for heterogeneous cointegrated panels.
     \newblock In B. Baltagi (Ed.), {\em Nonstationary Panels, Panel Cointegration, and Dynamic Panels: 15
        (Advances in Econometrics)}, 93--130. New York, NY: JAI Press.

     \bibitem[\protect\citeauthoryear{Pesaran and Pesaran}{Pesaran and
        Pesaran}{1993}]{PesaranPesaran}
        Pesaran, M.~H. and B.~Pesaran (1993).
     \newblock A simulation approach to the problem of computing {C}ox's
        statistic for testing nonnested models.
     \newblock {\em Journal of Econometrics\/} {\em 57}, 377--92.

     \bibitem[\protect\citeauthoryear{Phillips}{Phillips}{1991}]{Phillips}
        Phillips, P.~C.~B. (1991).
     \newblock To criticise the critics: an objective {B}ayesian
        analysis of stochastic trends.
     \newblock {\em Journal of Applied Econometrics\/} {\em 6}, 333--64.

     \bibitem[\protect\citeauthoryear{R Core Team}{R Core Team}{2012}]{RCoreTeam}
        R Core Team (2012).
     \newblock {\em R: A Language and Environment for Statistical Computing}.
     \newblock Vienna: R Foundation for Statistical Computing.

    \end{thebibliography}


  \subsubsection{Appendices}

Mathematical proofs that are an integral part of the paper's main narrative should be included in its main text, but all other proofs should appear in its appendices or as Supporting Information. More generally, \textbf{Supporting Information} can be used for proofs, subsidiary results, and
    their proofs. Numbering of equations should be appendix-specific. If more than a single
    appendix is required, appendices should be labelled A, B, etc., and items within the appendices (such as equations,
    figures and tables) should be labelled (A.1), Figure A.1 and Table A.1, etc. With an online Appendix
    in the Supporting Information, these items can be labelled   (S.1), Figure S.1 and Table S.1, etc.  \\

    \textbf{NB.} No subsidiary results or their proofs should be placed within another proof. \\

    Data sources should also be recorded in an appendix. \\

     \section*{Appendix A: Proofs of Results}
      \renewcommand{\theequation}{A.\arabic{equation}}
      \renewcommand{\thesection}{A}
      \setcounter{equation}{0}

      \medskip

      \textbf{Proof of Theorem~\ref{th-limdistr}:} For $\hat{g} _{i}=
       g_{i}(\hat{\beta}),$ by ... $\Box$

      Let
      \begin{equation}
       \pi _{i}(\beta )\equiv \frac{\rho _{1}(\hat{\lambda }
       (\beta )^{\prime}g _{i}(\beta ))}
       {\sum _{j=1}^{n}\rho _{1}(\hat{\lambda }
       (\beta )^{\prime}g _{j}(\beta ))},
       (i=1,...,n);  \label{empprob}
      \end{equation}
      ... \hfill$\square$\\

    \noindent was coded as

      \begin{verbatim}
       \section*{Appendix A: Proofs of Results}
       \renewcommand{\theequation}{A.\arabic{equation}}
       \renewcommand{\thesection}{A}
       \setcounter{equation}{0}

       \medskip

       \textbf{Proof of Theorem~\ref{th-limdistr}:} For $\hat{g} _{i}
        =g_{i}(\hat{\beta}),$ by ... $\square$ \\

       Let
       \begin{equation}
        \pi _{i}(\beta )\equiv \frac{\rho _{1}(\hat{\lambda }
        (\beta )^{\prime}g _{i}(\beta ))}
        {\sum _{j=1}^{n}\rho _{1}(\hat{\lambda }
        (\beta )^{\prime}g _{j}(\beta ))},
        (i=1,...,n);  \label{empprob}
       \end{equation}
       ... \hfill$\square$
      \end{verbatim}

    \textbf{NB.} Introductory statements such as, e.g., \textbf{Proof of Theorem 3.1:}, for proofs
     should only be used for results that appear in the main text. Where the proof immediately follows
     the statement of a lemma etc. in an Appendix only the introduction to the proof \textbf{Proof:}
     should be used.


    \renewcommand{\thesection}{2}


  \subsubsection{Supporting Information}

    Detailed simulation evidence, subsidiary results and their proofs plus any additional empirical results should be
     submitted as an online Appendix to be published as \textbf{Supporting Information}. If there is already an appendix or appendices to the paper,  equations, figures and tables can be referred to as (S.1), Figure S.1 and Table S.1, etc. \\

    \textbf{NB.} No subsidiary results or their proofs should be placed within another proof. \\

    \textbf{NB.} Introductory statements such as, e.g., \textbf{Proof of Theorem 3.1:}, for proofs
     should only be used for results that appear in the paper. Where the proof immediately follows
     the statement of a lemma etc. only the introduction to the proof \textbf{Proof:} should be used.


  \section*{Common Errors}

    This section details a number of errors often made which may result in a paper being returned to
    the author(s) for correction with a consequent delay in the editorial process. \\

      \begin{enumerate}
        \item[(a)] Failure to read the Submission Guidelines carefully. \\
        \item[(b)] Inclusion of detailed simulation results in the paper rather than in an online appendix to be published as Supporting Information. \\
        \item[(c)] Failure to include an empirical example in a theoretical paper. \\
        \item[(d)] Failure to number equations, theorems, remarks, etc., by section and to re-initialize
          the corresponding counters at the beginning of each section. \\
        \item[(e)] References:
          \begin{enumerate}
               \item[(i)] Omission of volume numbers for journal articles;
                \item[(ii)] Volume numbering in bold rather than italics;
                \item[(iii)] Omission of place of publication and publisher for books, texts, monographs, etc.;
                \item[(iv)] Omission of page numbers of chapters in books or articles in journals.
             \item[(v)] Inclusion of issue numbers for journal articles (with the exception of two journals previously named). \\
            \item[(vi)] Ordering of last names and initials in the references. \\
          \end{enumerate}
        \item[(f)] Text:
          \begin{enumerate}
            \item[(i)] Omission of commas in citations within parentheses `(Author, Year, Other, Year, and Another, Year)'. \\
            \item[(ii)] Failure to capitalize the first letters of Table and Figure in the text. \\
            \item[(iii)] Placing `et al.', `e.g.', `i.e.', etc., in italics rather than normal text. \\
            \item[(iv)] Failure to place footnote numbers at the end of the sentence after the full stop. \\
            \item[(v)] Placing footnote numbers within theorems, etc., rather than at the end of the  preceding sentence. \\
            \item[(vi)] Failure to abbreviate `for example' and `that is' within parentheses as e.g. and i.e. \\
            \item[(vii)] Placing the text of examples, remarks, etc., in italics rather than normal text. \\
          \end{enumerate}
        \item[(g)] Tables and Figures:
          \begin{enumerate}
            \item[(i)] Inclusion of notes to Tables in the caption. \\
            \item[(ii)] Placing captions of Tables and Figures in other than normal text.  \\
            \item[(iii)] Placing notes to Tables in other than \verb|\footnotesize| and with other
            than \verb|11pt| line spacing.
          \end{enumerate}
       \end{enumerate}


\end{document}
